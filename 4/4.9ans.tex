\paragraph{解:}
\begin{enumerate}
	\item 计算候选键:\\
	$C_{1}=\emptyset, C_{2}=\{C\}, C_{3}=\{A\}, C_{4}=\{B\}$ \\
	而$(C_{1} \cup C_{2})^{+}=U$,故\{C\}为唯一的候选键。\\
	因为A对于C有传递函数依赖,所以该关系模式最高属于第二范式。
	
	\item 计算候选键:\\
	$C_{1}=\emptyset, C_{2}=\{B\}, C_{3}=\{C, A\}, C_{4}=\emptyset$ \\
	而$(C_{1} \cup C_{2})^{+}=U$,故\{B\}为唯一的候选键。\\
	该关系模式最高属于BC范式。
	
	\item 计算候选键:\\
	$C_{1}=\emptyset, C_{2}=\{A\}, C_{3}=\{A\}, C_{4}=\{B, C\}$ \\
	而$(C_{1} \cup C_{2})^{+} \neq U$,再计算$(C_{1} \cup C_{2} \cup \{B\})^{+} \neq U$, 再计算$(C_{1} \cup C_{2} \cup \{C\})^{+}=U$故\{A, C\}为候选键。\\
	因为$B \to C$中B不包含候选键,所以该关系模式最高属于第三范式。
	
	\item 计算候选键:\\
	$C_{1}=\emptyset, C_{2}=\{A, B\}, C_{3}=\{C, D\}, C_{4}=\emptyset$ \\
	而$(C_{1} \cup C_{2})^{+}=U$,故\{A, B\}为唯一的候选键。\\
	该关系模式最高属于BC范式。
	
	\item 计算候选键:\\
	$C_{1}=\emptyset, C_{2}=\{A, B\}, C_{3}=\{C, D\}, C_{4}=\emptyset$ \\
	而$(C_{1} \cup C_{2})^{+}=U$,故\{A, B\}为唯一的候选键。\\
	因为$B \to D$中的非主属性D部分依赖于候选键,所以该关系模式最高属于第二范式。
\end{enumerate}