\documentclass[]{ctexart}
\usepackage{subfiles}
\usepackage{floatrow}
\usepackage{float}
\usepackage{graphicx}
\usepackage{amssymb}
\usepackage{amsmath}

%opening
\title{数据库系统原理 \\ 作业 4}
\author{软件42 \\ 欧阳鹏程 \\ 2141601030 \\ 版权声明:Creative Commons BY-NC-SA}

\begin{document}

\maketitle

\begin{enumerate}
	\item[4.1] 试根据下图所示关系判断以下函数依赖是否在$r_{1}$和$r_{2}$上成立:$A \to B, C \to A, CB \to A$。
	\begin{table}[H]
		\centering
		\caption{$r_{1}$}
		\begin{tabular}{|c|c|c|}
			\hline
			A & B & C \\\hline
			$a_{1}$ & $b_{1}$ & $c_{1}$ \\
			$a_{2}$ & $b_{1}$ & $c_{1}$ \\
			$a_{3}$ & $b_{2}$ & $c_{3}$ \\
			$a_{4}$ & $b_{4}$ & $c_{4}$ \\\hline
		\end{tabular}
	\end{table}

	\begin{table}[H]
		\centering
		\caption{$r_{2}$}
		\begin{tabular}{|c|c|c|}
			\hline
			A & B & C \\\hline
			$a_{1}$ & $b_{1}$ & $c_{1}$ \\
			$a_{1}$ & $b_{1}$ & $c_{2}$ \\
			$a_{2}$ & $b_{2}$ & $c_{2}$ \\
			$a_{4}$ & $b_{4}$ & $c_{3}$ \\\hline
		\end{tabular}
	\end{table}
	\subfile{4.1ans.tex}

	\item[4.3] 以下逻辑蕴涵是否成立?如果成立,请给予证明,否则请举出反例说明。
	\begin{enumerate}
		\item $\{AB \to C\} \vDash \{A \to C\}$
		\item $\{A \to BC, C\to DE\} \vDash \{A \to BCDE\}$
		\item $\{AB \to C, B \to C\} \vDash \{A \to C\}$
	\end{enumerate}
	\subfile{4.3ans.tex}

	\item[4.4] 关系模式R(A, B, C, D, E, G)上的函数依赖集合$F=\{A \to B, C \to G, E \to A, CE \to D\}$,试计算$(AG)^{+}_{F}$及$(CE)^{+}_{F}$。
	\subfile{4.4ans.tex}
	
	\item[4.5] 设有关系模式R(A, B, C, D, E, G, H),R上的函数依赖集合$F=\{CD \to B, CDE \to A, A \to B, B \to E, G \to AEH, H \to EG\}$:
	\begin{enumerate}
		\item F是最小函数依赖集吗?如果不是,试求出F的最小覆盖;
		\item 求R的所有候选键。
	\end{enumerate}
	\subfile{4.5ans.tex}

	\item[4.6] 设有关系模式R(A, B, C, D, E, G)及其函数依赖集合$F=\{B \to E, D \to G, A \to B, E \to A, DE \to C\}$,试判断R的一个分解$\rho=\{R_{1}(D, G), R_{2}(B, E), R_{3}(C, D, E), R_{4}(A, B)\}$是否是无损连接。
	\subfile{4.6ans.tex}
	
	\item[4.7] 设有关系模式R(A, B, C, D)及其函数依赖集合$F=\{C \to D, D \to A, C \to B, B \to A\}$:
	\begin{enumerate}
		\item 试判断R的一个分解$\rho=\{R_{1}(A, C), R_{2}(B, D), R_{3}(C, D)\}$是否是无损连接。
		\item 计算F在$\rho$中每个子关系模式上的投影,分解$\rho$保持函数依赖吗?
	\end{enumerate}
	\subfile{4.7ans.tex}

	\item[4.9] 以下关系模式最高属于第几范式?请简单解释原因:
	\begin{enumerate}
		\item $R_{1}(A, B, C), F=\{C \to B, B \to A\}$
		\item $R_{2}(A, B, C), F=\{B \to C, B \to A\}$
		\item $R_{3}(A, B, C), F=\{B \to C, AC \to B\}$
		\item $R_{4}(A, B, C, D), F=\{A \to C, B \to D\}$
		\item $R_{5}(A, B, C, D), F=\{B \to D, AB \to C\}$
	\end{enumerate}
	\subfile{4.9ans.tex}

	\item[4.11] 设有关系模式R(A, B, C, D,E, G)及其函数依赖集合$F=\{B \to C, D \to E, BD \to A, E \to G\}$:
	\begin{enumerate}
		\item 将R分解为3NF,且既无损连接又保持函数连接;
		\item 将R分解为BCNF,且保持无损连接。
	\end{enumerate}
	\subfile{4.11ans.tex}
\end{enumerate}

\end{document}
