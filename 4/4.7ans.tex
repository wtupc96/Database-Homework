\paragraph{解:}
\begin{enumerate}
	\item 初始符号表:
	\begin{table}[H]
		\centering
		\begin{tabular}{|c|c|c|c|c|}
			\hline
			T & A & B & C & D \\\hline
			$R_{1}$ & $a_{1}$ & $b_{12}$ & $a_{3}$ & $b_{14}$ \\
			$R_{2}$ & $b_{21}$ & $a_{2}$ & $b_{23}$ & $a_{4}$ \\
			$R_{3}$ & $b_{31}$ & $b_{32}$ & $a_{3}$ & $a_{4}$ \\\hline
		\end{tabular}
	\end{table}
	最终符号表为:
	\begin{table}[H]
		\centering
		\begin{tabular}{|c|c|c|c|c|}
			\hline
			T & A & B & C & D \\\hline
			$R_{1}$ & $a_{1}$ & $b_{12}$ & $a_{3}$ & $a_{4}$ \\
			$R_{2}$ & $a_{1}$ & $a_{2}$ & $b_{23}$ & $a_{4}$ \\
			$R_{3}$ & $a_{1}$ & $b_{12}$ & $a_{3}$ & $a_{4}$ \\\hline
		\end{tabular}
	\end{table}
	没有全a行,故该分解不是无损连接。
	
	\item $F_{1}=\{C \to A\}, F_{2}=\emptyset, F_{3}=\{C \to D\}$, \\ 令$G=\cup_{i=1}^{3}F_{i}=\{C \to A, C \to D\}$,得$X_{G}^{+}=\{A, D\}$,而$B \notin X_{G}^{+}$,故分解$\rho$不保持函数依赖。
\end{enumerate}