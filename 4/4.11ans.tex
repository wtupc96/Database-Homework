\paragraph{解:}
\begin{enumerate}
	\item 
	\begin{itemize}
		\item 先求最小覆盖F: \\
		\begin{itemize}
			\item 假设$B \to C$是多余的函数依赖,去掉$B \to C$得到$F_{1}=\{D \to E, BD \to A, E \to G\}$\\
			计算$(B)^{+}_{F_{1}}=\{B\}$,而$C \notin (B)^{+}_{F_{1}}$ \\
			所以$B \to C$不是多余的依赖关系,不能去掉。
			
			\item 假设$D \to E$是多余的函数依赖,去掉$B \to C$得到$F_{2}=\{B \to C, BD \to A, E \to G\}$\\
			计算$(D)^{+}_{F_{2}}=\{D\}$,而$E \notin (D)^{+}_{F_{2}}$ \\
			所以$D \to E$不是多余的依赖关系,不能去掉。
			
			\item 假设$BD \to A$是多余的函数依赖,去掉$B \to C$得到$F_{3}=\{B \to C, D \to E, E \to G\}$\\
			计算$(BD)^{+}_{F_{3}}=\{B, D, C, E, G\}$,而$C \notin (BD)^{+}_{F_{3}}$ \\
			所以$BD \to A$不是多余的依赖关系,不能去掉。	
		\end{itemize}
		同理,$E \to G$也不是多余的函数依赖。\\
		故最小覆盖$F=\{B \to C, D \to E, BD \to A, E \to G\}$。
		
		\item 求候选键:\\
		$C_{1}=\emptyset, C_{2}=\{B, D\}, C_{3}=\{C, A, G\}, C_{4}=\{E\}$ \\
		而$(C_{1} \cup C_{2})^{+}=U$,故\{B, D\}为唯一的候选键。
		
		\item 由于函数依赖没有公共的决定子,所以分得新的关系模式为:\\
		$\rho=\{R_{1}(B, C), R_{2}(D, E), R_{3}(B, D, A), R_{4}(E, G)\}$
		
		\item 观察可得由候选键构成的关系模式$R_{ck}$已经包含于$R_{3}$,所以$\rho$已经是分解完成的第三范式。
	\end{itemize}

	\item 
	\begin{itemize}
		\item R上的候选键是\{B, D\},而R上的依赖关系$B \to C, D \to E, E \to G$都不满足BCNF的要求;
		\item 首先选择$B \to C$作为分解对象,将R分解为$R_{1}(B, C)$和$R_{2}(A, B, D, E, G)$;
		\item 分解$R_{2}$:$R_{2}$上的候选键是\{B, D\},而$R_{2}$上的依赖关系$D \to E, E \to G$都不满足BCNF的要求;
		\item 分解$R_{2}$为:$R_{2}(D, E)$和$R_{3}(A, B, D, E, G)$;
		\item 分解$R_{3}$,$R_{3}$上的候选键是\{B, D, E\},而R上的依赖关系$BD \to A, E \to G$都不满足BCNF的要求;
		\item 分解$R_{3}$为:$R_{3}(B, D, A)$和$R_{4}(E, G)$;
		\item 至此,所有关系都满足BCNF,分解完毕。
	\end{itemize}
\end{enumerate}
