\documentclass[UTF8]{ctexart}

%opening
\title{数据库系统原理 \\ 作业 1}
\author{软件42 \\ 欧阳鹏程 \\ 2141601030}

\begin{document}

\maketitle

\begin{itemize}
	\item[1.3] 数据库系统在数据管理方面有什么重要特征?
	\paragraph{答:}特征如下:
	\begin{enumerate}
		\item 有良好的数据模型
		\item 数据共享性好
		\item 数据由DBMS统一管理:
		\begin{enumerate}
			\item 安全性
			\item 完整性
			\item 并发控制
			\item 故障恢复
		\end{enumerate}
	\end{enumerate}

	\item[1.6] 数据模型(data model)与数据模式(data schema)的区别和联系是什么?
	\paragraph{答:}
	数据模式是用数据模型描述数据时“型”的描述,不考虑值;数据模型是描述数据的手段,而数据模式是用给定的数据模型对某类具体数据的描述。
	
	\item[1.10] 数据库管理系统(DBMS)的主要功能有哪些?
	\paragraph{答:}主要功能如下:
	\begin{enumerate}
		\item 提供各类用户接口
		\item 数据目录管理
		\item 数据库的运行管理
	\end{enumerate}

	\item[1.14] 数据目录的基本功能和内容是什么?和一般数据相比,有哪些区别?
	\paragraph{答:}数据库目录是数据库服务器存放数据文件的地方,不仅包括有关表的文件,还包括数据文件和的服务器选项文件。不同的分发,数据库目录的缺省位置是不同的。数据目录也是对数据库结构进行定义及描述所得到的数据,其基本功能是对数据库的数据进行统一管理,实现大范围内的共享;和一般数据相比,数据目录只能由系统定义和为系统所有,在初始化时有系统自动生成,而不可能用SQL之类的语句定义,因为在数据目录还未定义时,任何SQL语句都无法执行,而一般数据可以用SQL之类的语句定义。
\end{itemize}

\end{document}
