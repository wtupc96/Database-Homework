\documentclass[]{ctexart}
\usepackage{subfiles}

%opening
\title{数据库系统原理 \\ 作业 3}
\author{软件42 \\ 欧阳鹏程 \\ 2141601030 \\ 版权声明:BY-NC-SA}

\begin{document}

\maketitle

\begin{enumerate}
	\item[3.2] 在题1中定义的基本表上完成以下查询:
	\begin{enumerate}
		\item 查询选修课程“CS-02”的学生学号、成绩。
		\item 查询选修课程“EE-01”的女学生姓名。
		\item 查询不选修课程“CS-02”的学生姓名。
		\item 查询身高高于“王涛”同学的男生学号、姓名及年龄。
		\item 查询选修课程“CS-01”的学生中成绩最高的学生学号。
		\item 查询学生姓名及其所选修课程的课程号、学号和成绩。
		\item 查询平均成绩超过80分的学生姓名和平均成绩。
		\item 查询选修三门及以上课程(包括三门)的学生已获得的学分数,并按学号进行升序排列。
	\end{enumerate}
	\subfile{3.2ans.tex}

	\item[3.3] 分别在Student和Course表中加入记录(‘01032005’,‘刘静’,‘男’,‘1983-12-10’,1.75,‘西14舍312’)及(‘CS-03’,‘离散数学’,64,4,‘陈建明’)。
	\subfile{3.3ans.tex}
	
	\item[3.4]
	将Student表中已修学分数大于110的学生记录删除。
	\subfile{3.4ans.tex}
	
	\item[3.5]
	将“张明”老师负责的“信号与系统”课程的学时数调整为56,同时增加一个学分。
	\subfile{3.5ans.tex}
	
	\item[3.7]
	在题3.1定义的基本表上建立如下视图:
	\begin{enumerate}
		\item 所有居住在“西18”舍的男生视图,包括学号、姓名、出生日期、身高等属性。
		\item “张明”老师所开设课程情况的视图,包括课程编号、课程名称、平均成绩等属性。
		\item 所有选修了“人工智能”课程的学生视图,包括学号、姓名、成绩等属性。
	\end{enumerate}
	\subfile{3.7ans.tex}
\end{enumerate}

\end{document}
