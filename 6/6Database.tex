\documentclass[]{ctexart}
\usepackage{subfiles}
\usepackage{floatrow}
\usepackage{float}
\usepackage{graphicx}
\usepackage{amssymb}
\usepackage{amsmath}
\usepackage[usenames,dvipsnames]{pstricks}
\usepackage{epsfig}
\usepackage{pst-grad} % For gradients
\usepackage{pst-plot} % For axes
\usepackage[space]{grffile} % For spaces in paths
\usepackage{etoolbox} % For spaces in paths
\makeatletter % For spaces in paths
\patchcmd\Gread@eps{\@inputcheck#1 }{\@inputcheck"#1"\relax}{}{}
\makeatother

%opening
\title{数据库系统原理 \\ 作业 6}
\author{软件42 \\ 欧阳鹏程 \\ 2141601030 \\ 版权声明:Creative Commons BY-NC-SA}

\begin{document}

\maketitle

\begin{enumerate}
	\item[7.3] 在例7-1中查询语句为:找出选修李明老师所教课程并且成绩为优的女生学号和姓名。
	\begin{enumerate}
		\item 试写出该查询相应的关系代数表达式。
		\item 画出表达式的语法树并对其进行优化。
	\end{enumerate}
	\subfile{7.3ans.tex}

	\item[7.4] 在例7-1中,关系Course在Cname属性上有有序索引,Teacher属性上有无序索引。实现下列选择运算的较优化方法是什么?
	\begin{enumerate}
		\item $\sigma_{Teacher='\text{李明}'}(E)$
		\item $\sigma_{Cname='\text{数据库系统}' and Teacher='\text{李明}'}(E)$
		\item $\sigma_{Cname='\text{数据库系统}' or Teacher='\text{李明}'}(E)$
	\end{enumerate}
	\subfile{7.4ans.tex}

	\item[7.5] 在习题7.3中,关系Course有20000个元组,块因子为20;关系SC有400个元组,块因子为10;假设内存被划分为6块,关系Course和SC的连接属性有序存储。对于$Course \bowtie SC$运算,试分析下列两种方法所需访问的物理块数。
	\begin{enumerate}
		\item 用嵌套循环连接算法。
		\item 用排序归并连接算法。
	\end{enumerate}
	\subfile{7.5ans.tex}
\end{enumerate}

\end{document}
