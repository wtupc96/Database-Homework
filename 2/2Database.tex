\documentclass[UTF8]{ctexart}
\usepackage{graphicx}
\usepackage{floatrow}
\usepackage{amsmath}
%opening
\title{数据库系统原理 \\ 作业 2}
\author{软件42 \\ 欧阳鹏程 \\ 2141601030}

\begin{document}

\maketitle

\begin{itemize}
	\item[2.3] 为什么要对关系模型加以完整性规则的限制?关系模型的完整性约束具体包括哪些内容?
	\paragraph{答:}完整性约束主要用于保证数据库中数据的正确性和相容性。关系模型的完整性约束包括:域完整性约束、实体完整性约束、参照完整性和用户定义完整性。
	
	\item[2.5] 试对笛卡儿积、$\theta$连接、等值连接、自然连接等关系运算进行比较。
	\paragraph{答:}首先连接运算($\theta$连接、等值连接、自然连接)都是建立在笛卡儿积运算之上的,比如$\theta$连接是对笛卡儿积做选择,而等值连接是$\theta$连接的一种特殊情况;自然连接是对笛卡儿积做投影。
	
	\begin{center}
		\begin{tabular}{|c|c|}
			\hline
			运算 & 公式 \\\hline
			笛卡儿积 & $R \times S = \{t^{(m+n)}| t^{(m) \in R \wedge t^{(n) \in S}}\}$ \\\hline
			
			$\theta$连接 & $R \bowtie_{A \theta B} S = \{rs|r \in R \wedge s \in S \wedge r[A] \theta s[B]\} = \sigma_{A \theta B}(R \times S)$ \\\hline
			
			等值连接 & $R \bowtie_{A = B} S = \{rs|r \in R \wedge s \in S \wedge r[A] = s[B]\} = \sigma_{A = B}(R \times S)$ \\\hline
			
			自然连接 & $R \bowtie S = \Pi_{Attr(R) \cup (Attr(S) - A)}(\sigma_{R.A = S.A}(R \times S))$ \\\hline
		\end{tabular}
	\end{center}
	
	\item[2.6] 试将$\cup, -, \bowtie, \sigma, \Pi$等运算转换为等价的元组关系演算形式,为什么要对关系演算表达式加以安全性约束?
	\paragraph{答:}
	\begin{itemize}
		\item $R \cup S = \{t|R(t) \vee S(t)\}$
		
		\item $R - S = \{t|R(t) \wedge \neg S(t)\}$
		
		\item 设R是m元关系的元组,S是n元关系的元组,R和S有p列表头一致(R元组的$m^{'}$ $\sim$ $m^{'} + p$和S元组的$n^{'}$ $\sim$ $n^{'} + p$的表头一致)
		\begin{equation}
			\begin{split}
				R \bowtie S  = & \{t^{(m+n-p)}|(\exists u^{(m)})(\exists v^{(n)})(R(u) \wedge S(v) \wedge \\ & \underbrace{u[m^{'}]=v[n^{'}] \wedge u[m^{'}+1] = v[n^{'}+1]\wedge \cdots \wedge u[m^{'}+p] = v[n^{'}+p]}_{p})
				\\ & \wedge t[1]=u[1] \wedge t[2]=u[2] \wedge \cdots \wedge t[m]=u[m] \wedge
				\\ & t[m^{'}+p+1]=v[n^{'}] \wedge \cdots t[m+n-p]=v[n]\}	
			\end{split}
		\end{equation}
		
		\item $ \sigma_{F}(R) = \{t|R(t) \wedge F^{'}\} $ \\ 其中$F^{'}$是选择条件F在元组关系演算中的等价表示。
		
		\item $\Pi_{i_{1},i_{2},\cdots,i_{k}}(R) = \{t^{(k)}|(\exists u)(R(u) \wedge t[1]=u[i_{1}] \wedge \cdots \wedge t[k]=u[i_{k}])\}$
		
		\item 因为在关系演算中,有可能会出现无限关系,进而出现无穷验证的问题;而无限关系和无穷验证在计算机中都是不允许出现的,所以为了保证运算的安全性需要采取安全性约束。
	\end{itemize}
	\item[2.7] 
	设有下列关系:
	\begin{table}
		\ttabbox{\caption{\textit{R, S, T}关系}}{
			\centering
			\begin{minipage}{0.35\textwidth}
				
				\begin{tabular}{|c|c|c|c|c|}\hline
					\textit{R} & \textit{A} & \textit{B} & \textit{C} & \textit{D}\\\hline					
					 & $a_{1}$ & $b_{1}$ & $c_{1}$ & $d_{1}$\\\hline	
					 			
					 & $a_{1}$ & $b_{1}$ & $c_{1}$ & $d_{2}$\\\hline	
					 					 
					 & $a_{2}$ & $b_{2}$ & $c_{2}$ & $d_{1}$\\\hline	
					 					 
					 & $a_{2}$ & $b_{3}$ & $c_{2}$ & $d_{2}$\\\hline	
					 					 
					 & $a_{2}$ & $b_{1}$ & $c_{2}$ & $d_{3}$\\\hline	
					 					 
					 & $a_{3}$ & $b_{2}$ & $c_{2}$ & $d_{1}$\\\hline	
					 					 
					 & $a_{3}$ & $b_{2}$ & $c_{3}$ & $d_{2}$\\\hline	
					 					 
					 & $a_{4}$ & $b_{3}$ & $c_{2}$ & $d_{1}$\\\hline	
					 					 
					 & $a_{4}$ & $b_{3}$ & $c_{2}$ & $d_{3}$\\\hline	
					 					 
					 & $a_{4}$ & $b_{1}$ & $c_{2}$ & $d_{4}$\\\hline	
					 				 
					 & $a_{4}$ & $b_{4}$ & $c_{2}$ & $d_{2}$\\\hline	
					  
				\end{tabular}
			\end{minipage}
			\hfil
			\begin{minipage}{0.35\textwidth}
				
				\begin{tabular}{|c|c|c|c|}       \hline
					\textit{S} & \textit{D} & \textit{E} & \textit{F}     \\\hline
					~ & $d_{1}$ & $e_{2}$ & $f_{1}$ \\\hline
					
					~ & $d_{2}$ & $e_{1}$ & $f_{2}$ \\\hline
					
					~ & $d_{2}$ & $e_{2}$ & $f_{3}$ \\\hline
					
					~ & $d_{3}$ & $e_{3}$ & $f_{1}$ \\\hline
				\end{tabular}
			\\
			\\
			\\
			
				\begin{tabular}{|c|c|c|c|}       \hline
					\textit{T} & \textit{D} & \textit{F} & \textit{G}     \\\hline
					~ & $d_{1}$ & $f_{2}$ & $g_{1}$ \\\hline
					
					~ & $d_{2}$ & $f_{2}$ & $g_{2}$ \\\hline
					
					~ & $d_{3}$ & $f_{1}$ & $g_{3}$ \\\hline
				\end{tabular}
			\end{minipage}			
	}
	\end{table}
	\begin{enumerate}
		\item[(1)] 求下列表达式的值:
		\newline
		$E_{1} = \Pi_{C, D}(\sigma_{A > 'a_{1}' \wedge B < 'b_{4}'}(R))$
		\newline
		$E_{2} = \Pi_{A, B, E, G}(\sigma_{A < 'a_{3}'\wedge E < 'e_{3}' \wedge G \neq 'g_{3}'}(R \bowtie S \bowtie T))$
		\newline
		$E_{3} = R \div \Pi_{D}(\sigma_{F = 'f_{1}'}(T))$
		\newline
		$E_{4} = \{t| (\exists u)(\exists v)(\exists w)(R(u) \wedge S(v) \wedge T(w) \wedge u[3] > 'c_{1}' \wedge v[2] \neq 'e_{2}' \wedge w[3] \neq 'g_{2}' \wedge u[4] = v[1] \wedge v[3] > w[2] \wedge t[1] = u[2] \wedge t[2] = u[3] \wedge t[3] = v[1] \wedge t[4] = w[3] \wedge t[5] = w[2])\}$
		\item[(2)] 试将$E_{4}$转化为等价的关系代数表达式。
	\end{enumerate}
	\paragraph{答:}
	\begin{enumerate}
		\item[(1)] $E_{1}$关系:
		\begin{table}[H]
			\caption{$R_{1}$}
			\begin{tabular}{|c|c|}
				\hline 
				$C$ & $D$ \\\hline
				
				$c_{2}$ & $d_{1}$\\\hline	
				
				$c_{2}$ & $d_{2}$\\\hline	
				
				$c_{2}$ & $d_{3}$\\\hline	
				
				$c_{2}$ & $d_{1}$\\\hline	
				
				$c_{3}$ & $d_{2}$\\\hline	
				
				$c_{2}$ & $d_{1}$\\\hline	
				
				$c_{2}$ & $d_{3}$\\\hline	
								
				$c_{2}$ & $d_{2}$\\\hline					
			\end{tabular}
		\end{table}
	
		$E_{2}$关系:
		\begin{table}[H]
			\begin{tabular}{|c|c|c|c|}
				\hline
				A & B & E & G \\\hline
				$a_{1}$ & $b_{1}$ & $e_{2}$ & $g_{1}$ \\\hline
				$a_{2}$ & $b_{2}$ & $e_{2}$ & $g_{1}$ \\\hline
				$a_{3}$ & $b_{2}$ & $e_{2}$ & $g_{1}$ \\\hline
				$a_{1}$ & $b_{1}$ & $e_{1}$ & $g_{2}$ \\\hline
				$a_{2}$ & $b_{3}$ & $e_{1}$ & $g_{2}$ \\\hline
				$a_{3}$ & $b_{2}$ & $e_{1}$ & $g_{2}$ \\\hline
			\end{tabular}
		\end{table}
	
		$E_{3}$关系:
		\begin{table}[H]
			\begin{tabular}{|c|c|c|}
				\hline
				A & B & C \\\hline
				$a_{4}$ & $b_{3}$ & $c_{2}$ \\\hline
			\end{tabular}
		\end{table}
		
	\end{enumerate}

	\item[2.8] 试用关系代数表达式写出在Student,Course、SC关系上进行的下列查询:
	\begin{enumerate}
		\item[(1)] 查询“计算机07”班同学的学号及姓名;
		\item[(2)] 学号为“01055107”的同学所选修的课程名称及成绩;
		\item[(3)] 未选修编号为“CS-05”课程的学生学号;
		\item[(4)] 选修了“张华”老师所开设课程的学生姓名、课程名称及成绩;
		\item[(5)] 选修了全部课程的学生姓名及班级。
		
		\paragraph{答:}
		\begin{enumerate}
			\item[(1)] $\Pi_{SNo, SName}(\sigma_{SClass='Computer 07'}(Student))$
			
			\item[(2)] $\Pi_{CName, Score}(\sigma_{SNo='01055107'}(SC) \bowtie Course)$
			
			\item[(3)] $\Pi_{SNo}(Student) - \Pi_{SNo}(\sigma_{CNo='CS-05'}(SC))$
			
			\item[(4)] $\Pi_{SName, CName, Score}(\sigma_{TName='ZhangHua'}(Course) \bowtie SC \bowtie Student)$
			
			\item[(5)] $\Pi_{SName, Class}(Student \bowtie (\Pi_{SNo, CNo} \div \Pi_{SNo}(Student)))$
		\end{enumerate}
	\end{enumerate}
\end{itemize}

\end{document}
