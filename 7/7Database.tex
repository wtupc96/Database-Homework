\documentclass[]{ctexart}

%opening
\title{数据库系统原理 \\ 作业 7}
\author{软件42 \\ 欧阳鹏程 \\ 2141601030 \\ 版权声明:Creative Commons BY-NC-SA}

\begin{document}

\maketitle

\begin{enumerate}
	\item[8.2] 在数据库系统实现中,有很多处理都可以看成由一个事务来实现。结合实际生活,试举一些例子。 

	\newpage
	\item[8.6] 数据转储的作用是什么?简单比较各种数据转储方式,并描述其适用的场合。

	\vspace{10cm}
	\item[8.9] 解释引入检查点方法的目的,并给出使用此方法进行恢复的步骤。
	
	\newpage
	\item[8.13] 比较串行调度和可串行调度,试举例说明二者有什么异同点。
	
	\vspace{10cm}
	\item[8.20] 死锁是什么?讨论预防与检测死锁的方法,并说明当发生死锁后如何恢复。
	
	\newpage
	\item[8.21] 有两个事务:\\
	$T_{1}$:时间标记为20 \\
	$T_{2}$:时间标记为30 \\
	如果对$T_{1}$,$T_{2}$对数据对象$R_{1}$,$R_{2}$按以下次序申请锁:\\
	$T_{1} Xlock R_{1}, T_{2} Xlock R_{2}, T_{1} Xlock R_{2}, T_{2} Xlock R_{1}$ \\
	请说明$T_{1}, T_{2}$在下列情况下的执行过程:
	\begin{enumerate}
		\item 一般的两段封锁。
		\item 具有wait-die策略的两段封锁。
		\item 具有wound-wait策略的两段封锁。
	\end{enumerate}
	如果$T_{1}, T_{2}$可以执行,则上述三种情况的等效串行执行次序如何?
\end{enumerate}

\end{document}
